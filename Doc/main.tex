\documentclass{Resources/netsci-project}
\usepackage{caption}
\usepackage{float}

% :::~ This is the configuration for the bibliography. DO NOT CHANGE
\usepackage[
    backend=biber,
    style=authoryear,
    natbib=false,
    maxcitenames=2,
    minbibnames=1, maxbibnames=99, 
    url=false, 
    doi=true,
    ]{biblatex}
    
    
\addbibresource{References/references.bib}

\subjectarea{Network Science}

\begin{document}
\firstpage{1}

\title{Final Project: Network scientific analysis about the vulnerability of the Swiss Railway network}
\author{Mathias Lüthi (15-707-037), Nino Scherrer (12-729-216), Peter Giger (14-915-383)}
\course{Network Science for Business, Economics, Informatics and Social Sciences}
\school{Faculty of Business, Economics and Informatics}
\date{09.12.2019}

\maketitle

\begin{abstract}
Railway networks are of great importance for every economy \autocite{Resilience}. Thus, it is all the more important to understand the reasons why and how they can fail. Yet, due to the size and complexity, their susceptibility to failures is not completely understood \autocite{Resilience}. This research project investigates the vulnerability of the Swiss railway network using network scientific models.
\end{abstract}


\section{Introduction}
Railway systems need to perform well even under suboptimal circumstances \autocite{Resilience}. Delays and cancellations are daily challenges for both passengers and railway companies \autocite{Resilience}. Existing research showed that the network topology has an effect on the performance of a railway system \autocite{ComplexTopology}. More recently, \textcite{Resilience} analyzed the resilience and robustness of a UK railway network. They showed that poor performance correlates more with cascade effects that failures \autocite{Resilience}.
\\~\\
Needless to say, the are many reasons why railway systems can fail. For this reason, the focus of this study is the Swiss railway system for which only few studies exist. In 2008, the authors \textcite{GraphSwiss} published a graph-theoretical analysis of the Swiss railway network over time. They've used measures such as degree/betweenness centrality to characterise the growth of the Railway network \autocite{GraphSwiss}. However, they've suggested further research to explain the robustness of the network. In 2009, the authors \textcite{VulnerabilitySwiss} proposed a generalized linear model (GLM) to assess the vulnerability of the Swiss road network. The downside of their approach is that they assumed the failures to be mutually exclusive \autocite{VulnerabilitySwiss}. 
\\~\\
This research projects extends the work of \textcite{GraphSwiss} and \textcite{VulnerabilitySwiss} using network scientific models. The primary focus is to analyze the Swiss railway network regarding graph characteristics and vulnerabilities.

\subsection{Transport Networks}
Transport networks are essential for people and economy \autocite{GraphSwiss} . Early research on transport networks focused on geometric and topological properties \autocite{GraphSwiss} . Later, with the availability of computational power, the research shifted towards network structures \autocite{GraphSwiss}. Recently, papers such as \textcite{Resilience} and \textcite{ComplexDelay} used graph properties to model robustness and delay dynamics.
\\~\\
There are several differences to other networks such as social networks \autocite{GraphSwiss}. Transport networks represent real objects such as lines and railway stations \autocite{GraphSwiss}. These physical objects have constraints themselves which influences the degree distribution \autocite{GraphSwiss}. Additionally, monetary constraints apply as well, thus restrict the ability for them to be scale-free \autocite{GraphSwiss}.

\subsection{The Swiss Railway Network}
According to the "European Railway Performance Index", Switzerland has one of the best performing railways  \autocite{RailwayPerformanceIndex}. Yet, the increased load over the past years is a constant challenge. Every delay or cancellation might result in cascade failures.
\\~\\
Every railway network is slightly different. This is due to their history and how they grew over the years \autocite{GraphSwiss}. The Swiss railway network is special because its early growth was purely driven by economic values \autocite{GraphSwiss}. This means that cities were prioritized and urban areas were not considered in the planning process \autocite{GraphSwiss}. Moreover, due to competition of private companies, the networks grew more or less independent of each other \autocite{GraphSwiss}. Today, most railways belong to the Swiss Federal Railways (SBB).

\section{Measures}
This section described the basic measures used in this research project.

\subsection{Degree Centrality}
The degree centrality is a fundamental measure based on the degree of the node \autocite{ComplexTopology}. The degree represents the number of connections a node has \autocite{ComplexTopology}. Hence, the degree centrality assumes that nodes with a large number of connections are more important \autocite{ComplexTopology}. It was first introduced by \textcite{Freeman} and \textcite{Freeman2}:
\begin{equation} \label{eqRestMass}
C_{i}^{D} = \dfrac{k_i}{N-1}
\end{equation}
where 
\begin{equation} \label{eqRestMass}
k_{i} = \sum_{j \in N}a_{ij}
\end{equation}
In railway networks, the degree centrality is constrained by spacial and economic limits \autocite{GraphSwiss}.

\subsection{Betweenness Centrality}
The betweenness centrality is the number of shortest paths passing through the node \autocite{ComplexTopology}. It was introduced  by \textcite{Freeman} and \textcite{Freeman2} as well:
\begin{equation} \label{eqRestMass}
C_{i}^{B} = \sum_{j \neq k} \dfrac{n_{jk}(i)}{n_{jk}}
\end{equation}
where, for the nodes $ j $ and $ k $, the shortest path is defined by $n_{jk}(i) $. In transportation networks, the betweenness centrality can be used to measure the impact of a node. \autocite{ComplexTopology}.

\subsection{Closeness Centrality}
The closeness centrality measures how close the node is to all other nodes \autocite{ComplexTopology}. It is therefore the inverse of the mean shortest path to all other nodes and defined as: \autocite{ComplexTopology}
\begin{equation} \label{eqRestMass}
C_{i}^{C} =\dfrac{1}{\sum_{j \neq i} d_{ij}}
\end{equation}
The closeness centrality depends on the geographic position of the nodes as well as the size of the network \autocite{GraphSwiss}. This restricts the comparability of different sized networks \autocite{GraphSwiss}..

\subsection{Clustering Coefficient}
The clustering coefficient measures the probability of two neighbours being connected \autocite{ComplexTopology}. It can therefore be defined as:
\begin{equation} \label{eqRestMass}
C_{i} =\dfrac{2m_i}{k_i(k_i -1)}
\end{equation}

\section{Data}
\subsection{Dataset}
The Dataset analyzed in this research project represents the Swiss Railway and Tram Network. It contains 3’190 railway and tram stations all over Switzerland with 3'353 connections between them. The data set we used was provided by geo.admin.ch the official geo information portal of the swiss confederacy. The data is from the 20.12.2017 and includes GIS information and data about the different public transport companies in Switzerland. \autocite{Dataset}
\\~\\
The data about the railway stations includes an id, a station name and geographic data, as shown in table \ref{tbl:RailwayStationData}. The connection data included an id, a start and end node, track operator, electrification, track width, distance and geographic data. All stations and connections mapped on Switzerland can be seen in figure \ref{fig:NetworkOnMap}.

\begin{table}[H]
\centering
\begin{tabular}{c c c }
\hline 
xtf-id & Betriebspunkt-Name &  geometry\\ 
\hline 
ch14uvag00092584 & Wabern, Eichholz & POINT (2600987.93 1197507.6371) \\
ch14uvag00092599 & Bern, Sandrain & POINT (2600494.741999999 1197843.540100001) \\
ch14uvag00092576 & Basel, Rheingasse & POINT (2611516.495999999 1267823.809999999) \\
ch14uvag00092992 & Zürich, Schiffbau & POINT (2681631.868000001 1249127.912) \\
ch14uvag00092591 & Bern, Kursaal & POINT (2600835.989999998 1200203.734000001) \\
\hline 
\end{tabular}
\caption{Railway station data}
\label{tbl:RailwayStationData}
\end{table}

\begin{center}
    \centering
    \includegraphics[width=300pt]{Resources/Network_on_map}
    \captionof{figure}{Railway and tram network mapped on Switzerland}
    \label{fig:NetworkOnMap}
\end{center}

\subsection{Data Preparation}
While creating the graph for the network, we assigned the names of the stations as well as their location coordinates to the nodes. After that we added all the edges connecting to their respective stations. A visualization of this network can be seen in figure \ref{fig:NetworkNoMap}.

\begin{center}
    \centering
    \includegraphics[width=300pt]{Resources/Network_no_map}
    \captionof{figure}{Railway and tram network of Switzerland not geographically mapped}
    \label{fig:NetworkNoMap}
\end{center}
\noindent
Before starting with our analysis some minor data cleaning and preparation had to be done. Some of the bigger stations are subdivided into several data points, for example the station "Interlaken Ost" is sectioned by platforms ("Platform 1-2", "Platform 3-4", "Platform 5-8"). Also, some tram and independently managed train routes are not connected to the main network because stations that are right next to each other used different nodes ("Bern SBB", "Bern RBS"). This leads to the existence of nodes, which are geographically located next to each other, but nevertheless are not connected. The network is divided in 47 connected components, from which the biggest connected component includes 1650 nodes, which is around 53\% of the whole network. There are four nodes, which are not connected to any other node in the network. These nodes were excluded for the analysis of the network.
\\~\\
Because the network is divided in many connected components and the existence of close but not connected nodes, we decided to make some modifications to the network. Nodes whose spatial distance was less than 300 meters and no path between them existed, were merged to one node that includes the connections of all the merged nodes. This modification relies on the assumption that these nodes belong to the same station or are at least located in close walking distance. This results in a network with 3'146 nodes and 3'351 edges. The number of connected components is reduced to five, from which the giant connected component contains 99\% of all nodes in the network (3’116/3’146). This new better-connected network can be seen in figure \ref{fig:NetworkCleanedOnMap}.

\begin{center}
    \centering
    \includegraphics[width=300pt]{Resources/Network_cleaned_on_map}
    \captionof{figure}{Modified railway network with only 5 connected components}
    \label{fig:NetworkCleanedOnMap}
\end{center}


\subsection{Characteristics of the Network}
The emergence of the Railway network characterizes its degree distribution. There are few nodes with a relatively high degree, which represent the biggest traffic hubs, they are located in the economically most important cities of Switzerland (e.q. Basel, Bern, Geneva and Zurich). Most nodes do have only two connections to other stations. This follows the characteristic that most railway lines are going through stations in a sequenced fashion, with some start and end nodes with a degree of one and some connecting stations with a higher degree. The degree distribution can be seen in figure \ref{fig:DegreeDistribution}. This also explains the low clustering discovered in the network ($ C = 0.016 $). This distribution results in an average degree close to two ($ <k> = 2.13 $). The assortativity is 0.1678.

\begin{center}
    \centering
    \includegraphics[width=300pt]{Resources/degree_distribution}
    \captionof{figure}{Degree distribution of the swiss railway network}
    \label{fig:DegreeDistribution}
\end{center}


\section{Conclusion}
The Swiss Railway network is important to the Swiss citizens as well as the Swiss economy. Yet, increasing demand and tight schedules are a constant challence. However, there is a limit to the network especially because few new lines are added. This means that it is all the more important to understand how delays or failures influence the network.
\\~\\
This research project investigated the vulnerability of the Swiss Railway network. 
\\~\\
An appropriate countermeasure could be to split important stations into independent sub-networks such that a failure does not propagate through the whole network. 
\\~\\
This research project does not cover...
\\~\\
It remains to be further investigated...

\printbibliography

\end{document}
